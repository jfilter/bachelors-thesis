\documentclass[11pt]{article}

\usepackage[english]{babel}
\usepackage{blindtext}

\title{Interactive Visualization\\ of Large Concept Lattices\\for Exploratory Search}


\date{\today}
\author{Johannes Filter}

\begin{document}
\maketitle

\tableofcontents
\newpage

\section{Introduction}

The digital revolution is affecting every part of our life. Also the humanities scholars stand before a big change in their ways when huge analog collections are digitized. They have to apply computer science methods to organize and analyze huge amount of data. The term "Digital Humanities" evolved during the last 10 years which can be defined as an "intersection between the humanities and information technology" \cite{Svensson2010}.

 The Information Retrieval Department of the Universidad Nacional de Educación a Distancia (UNED) in Madrid (Spain) cooperates with human scholars to conduct research in the Digital Humanities. In this project, there are historical maps which have been digitized and annotated. To extract knowledge from the collection the research group advocates for the use of Formal Concept Analysis (FCA) for topic organization. \cite{Castellanos,Cigarran}
 
 They successfully implemented a FCA algorithm but lack an interactive user interface, which will be developed in this thesis. 
 
 The work will be evaluated by a small user study conducted at the UNED with users from different backgrounds.
 
\section{Background}

\subsection{Formal Concept Analysis}

Formal Concept Analysis (FCA) \cite{Ganter2012} is a mathematically well-funded technique to find relationships among objects and has been applied to areas\cite{Poelmans2013}.

Formally it is defined as as a tripple $K = (G, M, I)$ where G is a set of objects, M is a set of attributes and I is a binary relation $G \times M \rightarrow (0, 1)$. I specifies if an object has an attribute or not. Example: we have tree documents(letters) and our attributes categories if a word occurs in this document or not.

\begin{center}
\begin{tabular}{ l | c c c c}

  & Letter1 & Letter2 & Letter3 & Letter4\\
\hline

Love & 0 & 0 & 1 & 1\\
Hello & 1 & 0 & 1 & 0\\
Apple & 0 & 1 & 1 & 1\\
You & 1 & 0 & 0 & 1\\

\end{tabular}
\end{center}

HERE WILL BE AN IMAGE


Comments from Angel:


"It is a little bit incomplete and messy.

What you have in the table is a "Formal Context"

When you apply FCA, you obtain a set of "formal concepts", where a formal concept (A,B) is the maximum set of objects (A) sharing all the attributes in B.

These formal concepts can be organized by means of a "partial order relationship"

The set of formal concepts plus the order relationship results in a concept lattice which can be represented as a Hasse Diagram
"

Because of the ordered sets, they are a concept lattices. Orders sets can be represented as a Hasse Diagram. \cite{Ganter2012}

\subsection{Interface Design Principles}

Well-funded search interface design principles and principles of information visualization will be applied.\cite{Hearst2009,Shneiderman1996}

\blindtext

\subsection{Explorative Search}

\blindtext

\section{Related Work}

While some user studies proclaim that non-experts can read Hasse diagram\cite{Eklund2004}, the study has been conducted on relatively small lattices with a small amount of objects and attributes. It is easy to see, that the visual representation is hindern when the number of concepts grow. The amount and the edge crossings make understanding impossible. "Representing concept lattices constructed from large contexts often results in heavy, complex diagrams that can be impractical to handle and, eventually, to make sense of." \cite{Kuznetsov20072}

There exists basically several approaches to reduce complexity of large concept lattices. I describe in the following.
\subsection{Pruning Nodes}

A prominent approach to prune lattices called "iceberg lattice" \cite{Stumme2002}. A variation from the frequent item-set mining which specific min-support and min-confidence\cite{Agrawal1993}. It creates a top of the lattice but has some drawbacks because "One should be careful not to overlook small but interesting groups, for example, “exotic” or “emergent” groups not yet represented by a large number of objects, or, groups that contain objects who are not members of any other group."\cite{Kuznetsov20072} The iceberg lattice just focuses on the concepts that contain a lot of documents. That is why an other approaches exist: Stability\cite{Kuznetsov2007}: "A concept is stable if its intent does not depend much on each particular object of the extent." \cite{Kuznetsov20072}

	It is also possible to apply traditional cluster techniques to FCA.\cite{AswaniKumar2010}

\subsection{Local View}

There is been some research done to convey the insights of FCA without visualizing the lattice and just focusing on a local view. Sometimes referes as 'Conceptual Neighborhood'\cite{Eklund2009,Eklund2012}, you focus on one concept and show concepts that are directly adjacent to the concept, 'above' and 'below', in the lattice. The workflow is by initialize a concept with the help of a query interface and that to navigate incrementally by removing terms (going up) and adding terms (going  down). This approach has been applied to different scenarios. For instance for a 'Digital Museum' \cite{Eklund2009,Eklund2012}, image browsing \cite{Ducrou2006,Ducrou2008} and search engines \cite{Dau2008,Kim2006}.

There exist small variations: \cite{Carpineto1995} you can bound the search space by limiting the degree of parents, children or siblings to display. So to specify the maximum distance in a lattice to which nodes are displayed. It is also possible to make use of a fisheye view where  the nodes in the center of the screen are bigger and the nodes fade out of the screen close to the border

The CREDO system \cite{Carpineto2004} is similar to the Conceptual Neighborhood, you have to formulate a query fist and then a sub lattices is build. But there are also results included that do not directly adjacent. Calculated with a new similar measurement.

Small Variation: Transform to Tree

While transforming the lattice into a tree sounds promising, because you could apply sophisticated tree visualizing techniques to reduce edge crossing, it comes with several drawbacks. One naiv approach\cite{carpineto2004concept}: Duplicate nodes with more than one parents. While this is a successful transformation, we added a high amount of redundant nodes.

Another approach: just select the 'best' parents and hides all other edges to the other parents\cite{Melo2011}. While this technically not breaks the concept, the visual representation does not correspond to the underlying model.

This idea needs some more research to find out how useful it is.

\section{Fancy FCA 1.0}

\subsection{Idea}

Comments from Angel:
"There is another theoretical problem related to iceberg lattices.

FCA in general, and our work in particular, is intended to be an exploratory technique. That is, I know nothing (or almost nothing) about the data, so I would like to have a data organization (as the one provided by FCA in the concept lattice) that allows me to EXPLORE the data in order to gain some insight about.

When pruning the nodes, you are losing many data relationships, many formal concepts and, consequently, the "power" of FCA as exploratory technique is significantly reduced."

Because of huge amount of concepts in our project, a pruned version of a galois lattice can give a good overview but is not suitable for in-depth analysis.

I will develop an interface that comprises two different views: The Local View with an query extension to browse through the lattice and a pruned, global view to give an overview over the lattice.

As I remark in the state of the art review, pruning-based techniques suffer from .... On the other hand, when you apply local view you can lose the big picture..... To address both problems, we propose an hydrid approach integrating both representations in order to be able to offer a manageable visualization without losing the general view....

\subsection{Implementation}

The interface should run in modern Web Browser. That is why I utilize mainly the technologies: SVG and Javascript. I use the Javascript Framework d3.js for the visualizing besides several other modern web technologiy tools and frameworks. d3.js gives a good abstraction to most occures problems

The data sources comprises two APIs: 1. a precomputed concept lattices, 2. detailed information of a document, specified by a document id.

For an analysis of users, the user behaviour should be logged. That is why an backend has to be implemented  offering user registration. The logged can be saved in a relation database. An admin interface is not needed.

\section{First Evaluation}

\blindtext

\section{Fancy FCA 2.0}

\section{Second Evaluations}

\blindtext

\section{Conclusions}

\blindtext

\newpage

\bibliographystyle{plain}
\bibliography{biblography}
\end{document}